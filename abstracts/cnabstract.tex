\begin{abstract}

一个能理解、生成自然语言并能用自然语言流利交流的系统将会有广泛的用途,但目前的计算语言学软件几乎在所有关键领域的表现,都与人类的需求相差甚远。本文在自然语言理解、生成和对话系统方面展开了深入研究,旨在搭建一个以能实现智能对话为长远目标的自然语言处理工具集。

本文的工作主要在认知体系结构OpenCog上进行的,OpenCog是一个旨在实现通用人工智能(AGI)的集成软件平台,采用基于加权有向超图的知识表示体系Atomspace,不仅为自然语言过度到逻辑语义形式提供了平台,而且方便了常识推理和自适应学习机制等研究工作的开展。

在理论层面上,本文的核心研究问题是设计一个面向英语的智能会话系统,其中包含自然语言理解(将英语句子转换成逻辑表达形式)、自然语言生成(将逻辑表达式转换成英语句子)、面向超图的推理以及目标驱动的对话控制。本文对该设计的实现和相关的实验工作主要集中在自然语言理解和生成方面,使其成功地从设计阶段走向能实现很多实用功能的阶段。在推理和对话控制方面,我们也进行了相关实验,并搭建了一个能集成自然语言理解、生成、推理以及对话控制等模块的原型系统。

在技术层面上,本文的研究工作在以下四个假设中展开:1)借助一个超图转换系统,使用依存关系语法、传统逻辑与谓词逻辑的合理结合,将各式各样的自然语言表达转换成满足下列要求的逻辑表达方式,是可行的。2)借助由超图转换表示的推理规则和基于超图表示的知识库,使用简单的逻辑推理,在上述自然语言理解框架输出的逻辑表达式上实现基本的常识推理,是可行的。3)借助一个超图转换系统和一个超图匹配系统,在由自然语言理解系统自动生成的二元组(语言表达式,逻辑表达式)组成的知识库中根据逻辑表达式找到匹配的语言表达式并生成自然语言,是可行的。4)利用上述的语言理解、生成和逻辑推理系统,构建一个有用且灵活的智能会话系统,是可行的。

在自然语言理解方面,本文论证了一个可行的自然语言理解流水线,它集成了改进后的卡耐基-梅隆大学的链语法解析器、改进后的RelEx关系抽取系统以及一个全新的RelEx2Logic系统,能将RelEx的输出转换成OpenCog所使用的概率逻辑网络形式的逻辑表达式。在自然语言生成方面,本文在OpenCog系统内实现了一个全新的自然语言生成系统,通过微观规划和表层生成,能将一个给定的逻辑表达式转换成相应的英文句子。在基于语言的推理方面,借助于OpenCog中的概念逻辑网络(PLN)框架,以基于上述自然语言理解框架输出的逻辑表达式作为推理前提,本文实现了不同形式的逻辑推理实例。最后,本文论证了上述的集成系统框架能被应用在自然语言问题系统和智能会话系统中。

本文阐述的研究工作目前仍在继续,距离达到人类水平的智能会话系统的长远目标,还有很长的路要走。但是,本文的理论研究工作和相关的实用工具的开发,已经向着我们的长远目标迈出了一大步,无论是被应用在现实世界上的相关自然语言软件,还是在OpenCog的推理系统上实现的基于自然语言的常识推理,都在自然语言处理领域有着深远的意义。

\keywords{自然语言理解;自然语言生成;知识表示;对话系统}
\end{abstract}
